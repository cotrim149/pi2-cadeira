\chapter[Introdução]{Introdução}
\section{Contextualização}


A má educação alimentar é uma realidade no contexto brasileiro. Esta situação é
refletida nas estatísticas feitas pela VIGITEL(Vigilância de Fatores de Risco e Proteção
para Doenças Crônicas por Inquérito Telefônico) em 2013, que segundo a pesquisa,
50,8\% da população brasileira estão acima do peso ideal. Adicionalmente, nesse grupo,
tem-se ainda que 17,5\% são obesos.

Ainda considerando a questão da educação alimentar, indicam-se consultas a
profissionais, normalmente nutricionistas, visando obter recomendações quanto à
adequada dieta alimentar, dado o perfil do interessado e as condições para que este se
mantenha saudável. Apesar dessa consulta ao profissional ser adequada, outras formas
de acesso à recomendação alimentar poderiam permitir aos interessados adquirir
conhecimentos sobre a importância de uma dieta alimentar balanceada para manter uma
vida saudável.

Diante do exposto e dado que hoje temos a presença em massa de dispositivos
móveis no cotidiano das pessoas, estudos sobre o uso desses dispositivos como meios
de prover informações quanto à reeducação alimentar tornam-se relevantes, ampliando a
difusão do conhecimento sobre boas práticas alimentares.

Entretanto, cabem alguns cuidados nessa abordagem de facilidade de acesso,
dentre eles: (i) as informações providas pelo aplicativo no dispositivo móvel devem ter
sido validadas com profissionais devidamente capacitados da área médica, no caso
nutricionistas, psicólogos e outros; (ii) as limitações de capacidade de memória e
processamento, intrínsecas nos dispositivos móveis, devem ser contornadas com base no
projeto e na análise de algoritmos específicos para o fim proposto, e (iii) os estudos
quanto ao desempenho, à qualidade das recomendações, à usabilidade e demais critérios
de qualidade associados ao contexto em questão demandam esforços por parte dos engenheiros de software.

O aplicativo OnFit \cite{onfit} é um exemplo de iniciativa nesse contexto. Esse
aplicativo foi proposto pelos desenvolvedores Lucas Andrade, Thiago Bernardes e
Victor Cotrim (autor desse TCC), todos associados ao projeto BEPiD \cite{bepid}, ao longo do
período compreendido entre 30 de Março de 2014 e 27 de Março de 2015. Como foi
dito anteriormente, alguns cuidados devem ser tomados quanto a facilidade de acesso,
sendo um destes a equipe médica que está sendo usada como suporte especializado. No caso do aplicativo, essa equipe foi formada por: uma nutricionista, uma endocrinologista e um médico orto-molecular.

Atualmente, o aplicativo OnFit encontra-se em sua primeira versão, disponível
na App Store. Apesar do aplicativo ter sido disponibilizado para uso, o mesmo ainda
demanda refinamentos significativos quanto às questões de: (i) limitações de capacidade
de memória e processamento, e (ii) qualidade no que tange o desempenho, a adequação
das recomendações alimentares, a usabilidade e outros critérios de qualidade.

\section{Questão de pesquisa}

A análise dos alimentos a serem consumidos, quando feita por um nutricionista,
sempre leva em conta os seus macro e micro nutrientes \cite{entrevista_aline}. Essa análise nem sempre é
facilmente obtida, uma vez que o(a) nutricionista deverá considerar, dentre outros
aspectos do perfil do paciente: sexo, IMC(Índice de Massa Corporeo), quantidade de
dias no qual realiza as atividades físicas durante a semana, e o objetivo do paciente –
emagrecimento, ganho de massa ou manutenção de peso.

Ainda é pretendido, se possível em tempo hábil,  adaptar a solução proposta sob o ponto de vista dos nutricionistas, permitindo aos mesmos a análise da recomendação de alimentos considerando especificamente cada interessado.Nesse cenário, seria possível o nutricionista analisar e alterar os alimentos em
quantidade, e substituir esses últimos por outros alimentos com propriedades alimentares
semelhantes, caso haja necessidade.

Portanto, o trabalho proposto procurará responder a seguinte questão de pesquisa: É possível desenvolver um algoritmo capaz de auxiliar na recomendação de alimentos, levando em conta o perfil do interessado, a qualidade da dieta, questões de desempenho e demais critérios de qualidade relevantes para acesso em dispositivos móveis? E se possível, em tempo hábil, é possível adaptar o algoritmo de auxílio na recomendação de alimentos sob o ponto de vista dos nutricionistas, auxiliando-os na análise dos alimentos recomendados?

\section{Justificativa}

Considerando estudos quanto à dieta alimentar dos Brasileiros, tem-se que as
recomendações devem se basear em alimentos mais do que em nutrientes \cite{art_alimentacao_saudavel} . O
Ministério da Saúde, em 2014, disponibilizou um guia alimentar \cite{guia_alimentar} no intuito de
orientar a população Brasileira sobre alimentação saudável e balanceada.

Esse guia informa sobre a necessidade de uma reeducação alimentar no âmbito
nacional, salientando a importância de pesquisas nessa área e procurando ampliar as vias
de recomendação de alimentos. Outro fator que corrobora com essa necessidade é a
questão da obesidade sendo vista como uma epidemia, não apenas no Brasil, também no
mundo \cite{art_obesidade}.

Nesse cenário, pretende-se, com a proposta aqui apresentada, refinar o aplicativo
OnFit, visando adequadas recomendações de alimentos aos interessados, via
dispositivos móveis, e alinhadas aos objetivos específicos destes interessados -
emagrecimento, ganho de massa ou manutenção de peso.

\section{Objetivos}

Seguem os objetivos, geral e específico, da proposta.
\subsection{Geral}

Projetar e analisar algoritmos específicos para recomendação de alimentos,
usando uma abordagem orientada a critérios de qualidade, na qual destacam-se estudos
quanto ao desempenho em dispositivos móveis, qualidade da dieta recomendada e
usabilidade do aplicativo.

\subsection{Específicos}

Investigar possíveis soluções candidatas para viabilizar a recomendação de
alimentos, levando em conta o perfil e os objetivos desejados pelo interessado brm como considerando
questões de desempenho e usabilidade em dispositivos móveis;

Realizar experimentos para seleção de uma solução adequada à questão de
pesquisa, considerando as soluções candidatas identificadas no objetivo anterior.
Possibilidade de uso da Engenharia Experimental bem como da Análise de
Complexidade de Algoritmos na validação e verificação da pertinência da
solução;

Documentar essa solução, gerando artefatos orientados às boas práticas da
Engenharia de Software em diferentes níveis de abstração: requisitos, projeto,
codificação e teste.

