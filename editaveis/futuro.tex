\chapter[Futuro]{Futuro}

\section{Estrutura}

\subsubsection{Modelagem para a análise de tensões da estrutura}

O próximo passo é fazer uma análise computacional das tensões (pelo critério de falha de von Mises) equivalentes atuantes na estrutura, para a análise  será utilizado o software ANSYS. Após a modelagem ser feita teremos as tensões na estrutura, pelo método de elementos finitos, com isso os matérias para construção da estrutura serão escolhidos de acordo com a  ABNT 6061- T6.

\section{Power train}

\subsection{Motor}

Foram levantado possíveis motores para o projeto, o próximo passo é o levantamento de custos.

\begin{itemize}
 \item BOSCH GPD 12V 300W
 \item Motor de arranque:
  \begin{itemize}
    \item Zm 8070501 (12V 0,8 kW)
    \item Zm 8070502 (12V 0,8 kW)
    \item Zm 8010603 (12V 0,8 kW)
    \item Zm 8010604 (12V 0,8 kW)
    \item Zm 8010605 (12V 0,8 kW)
    \item Zm 8010606 (12V 0,8 kW)
    \item Zm 8010607 (12V 0,8 kW)
    \item Zm 8010608 (12V 0,8 kW)
  \end{itemize}
\end{itemize}

\section{Controle}

Este documento descreve o que foi elaborado como pesquisas iniciais. Para a continuidade do projeto foram definidos os  seguintes estudos no que tange aos tópicos de controle:

\begin{itemize}
 \item Prova de conceito para comunicação VPN e Bluetooth para iOS;
 \item Prova de conceito para comunicação VPN e Bluetooth para Android;
 \item Prova de conceito para comunicação de Joystick ( cabo e Bluetooth);
 \item Definição de interface entre os componentes de controle;
 \item Versão inicial do controle final com smartphone;
 \item Versão inicial do controle final com joystick.
\end{itemize}
