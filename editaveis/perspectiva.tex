\chapter[Perspectivas]{Perspectivas}

Este capítulo divide-se em duas seções, o incremento atual do produto e os próximos passos, ambas as etapas são divididas por área de desenvolvimento: estrutura, \textit{Power Train} e controle.

Na seção \textit{Incremento Atual} são levantadas as características e funcionalidades atuais do produto, já em \textit{Próximos Passos} são discutidas próximas tarefas e desenvolvimento e perspectivas do grupo quanto ao final do projeto.

\section{Incremento atual}
  \subsection{Estrutura}
    \begin{enumerate}
      \item Imagem da estrutura  atualizada e acoplada na cadeira de rodas(trabalho escrito);
      \item Melhora do acoplamento inferior – mudar o tipo de acoplamento minimizando o tempo de encaixe na cadeira;
      \item Alteração do acoplamento superior para torna-lo mais prático e diminuir o tempo de acoplagem;
      \item Melhoria no acoplamento inferior – aumento dos furos da parte traseira do dispositivo para que este pudesse se acoplar em mais cadeiras.
      \item Pintura da estrutura;
      \item Estudo de forças diante conceitos de von mises e centro de massa no trabalho escrita;
    	\item Melhoria dos parafusos com adaptação de borboletas para facilitar na rosca, e, assim, evitar o uso de ferramentas.
    \end{enumerate}
  \subsection{\textit{Power Train}}
    % \begin{enumerate}
    %   \item Escolha do motor elétrico, levando em consideração, os estudo matemáticos realizados e seu custo financeiro.
    %   \item Especificação do motor elétrico de Corrente Contínua quanto a potência minima;
    %   \item Estudo e especificação sobre baterias a serem utilizadas, quanto tensão, capacidade e dimensão. Foi escolhido usar a bateria de Chumbo-Ácida selada, levando em conta vantagens e limitações. Os cálculos de autonomia foram feitos para melhor estimar o método de carregamento da bateria;
    %
    % 	\item Estudo com o auxílio do \textit{\textit{software}} Matlab da interferência da roda e do peso do sistema no desempenho do conjunto de moto-redução;
    %   \item Os motores com redutor escolhido para o projeto possui potência de 305W e redução de 1:10;
    %
    % \end{enumerate}
    Não houve incremento relacionado a esta área.
    \subsection{Controle}
      \begin{enumerate}
        \item Mudança nas trilhas das pontes H para trilhas mais espessas de modo que os componentes da ponte H consigam aguentar a corrente que passa neles.
         \item Confecção do Joystick para usuário.
         \item Implementação de arquitetura paralela funcional.
         \item Implementação de 2 pulsos de PWM em somente 1 microcontrolador.
         \item Dissipadores de calor foram dimensionados para os transistores da ponte H, que podem chegar a altas temperaturas, as quais podem prejudicar seus componentes.
         \item Modelagem dos estados de movimentação da cadeira de rodas em relação ao \textit{joystick} e intensidade e direção dos motores.
       % \item Especificação de tecnologia: Estudo e escolha da melhor tecnologia voltada para o problema, que no caso será feito com um Raspberry Pi para comunicação entre a interface do usuário e o motor;
        %
        % \item Controle de potência: Estudo de qual tipo de controlador de potência a ser usado no motor, para controle de sua corrente. A escolha da utilização da ponte H será combinado com o algoritmo de PWM, que utiliza o Raspberry Pi como forma de resposta aos comandos dos usuários como a diração, aceleração, frenagem da cadeira motorizada adaptada;
        %
        % \item Escolha da linguagem de programação: Definição de qual linguagem será utilizada com base na problemática existente e nos recursos a serem utilizados. A linguagem escolhida foi o Phyton, pois existem bibliotecas especificas que controlam GPIO do Raspebery Pi.

      \end{enumerate}

  \subsection{Interface com Usuário}
    \begin{enumerate}
      \item Gerar um interface com o usuário utilizando o joystick para o controle da cadeira de rodas. Gerar, utilizando um display, indicadores que são importantes para o usuário, tais como: Velocidade e indicador da bateria.
    \end{enumerate}

\section{Próximos passos}
  \subsection{Estrutura}
  \subsection{\textit{Power Train}}
  \subsection{Interface com Usuário}
  \subsection{Controle}
