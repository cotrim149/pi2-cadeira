\chapter[Perspectiva]{Perspectiva}

\section{Incremento atual}
  \subsection{Estrutura}
    \begin{enumerate}
      \item Modelagem da cadeira motorizada adaptada: Foi feito a modelagem da cadeira de rodas levando em conta a solução de portabilidade e acessibilidade da mesma, pontuando pontos como o design de inovação do anexo autômato a cadeira.

      \item Escolha do material a ser usado em anexo: Um estudo dos possíveis materiais foi feito, alavancando os motivos e vantagens do uso do mesmo para suporte das cargas do anexo.

      \item Ergonomia do produto: Estudo e modelagem do melhor design da cadeira de rodas manual com o anexo que a motoriza.
    \end{enumerate}
  \subsection{Power Train}
    \begin{enumerate}
      \item NECESSARIO TERMINAR DE PEGAR AS INFORMAÇÕES DO GRUPO
      \item Especificação do motor a ser usado: Escolha de qual tipo do motor a ser utilizado na cadeira de rodas, levando em conta cálculos matemáticos como embasamento teórico.

      \item Pesquisa do tipo de bateria : Escolha do tipo de bateria, levando em conta o tipo, as vantegens e limitações.

      \item Especificação da bateria (Creio q seria melhor dimensionamento): Foi montada uma estratégia de carregamento(mudar esse nome) de bateria e um estudo da autonomia da mesma levando em conta cálculos matemáticos.
    \end{enumerate}

    \subsection{Controle}
      \begin{enumerate}
        \item Especificação de tecnologia: Estudo e escolha da melhor tecnologia voltada para o problema, que no caso será feito com um Raspberry Pi para comunicação entre a interface do usuário e o motor.

        \item Controle de potência: Estudo de qual tipo de controlador de potência a ser usado no motor, para controle de sua corrente. A escolha da utilização da ponte H será combinado com o algoritmo de PWM, que utiliza o Raspberry Pi como forma de resposta aos comandos dos usuários como a diração, aceleração, frenagem da cadeira motorizada adaptada.

        \item Escolha da linguagem de programação: Definição de qual linguagem será utilizada com base na problemática existente e nos recursos a serem utilizados. A linguagem escolhida foi o Phyton, pois existem bibliotecas especificas que controlam GPIO do Raspebery Pi.

      \end{enumerate}

  \subsection{Interface com Usuário}
    \begin{enumerate}
      \item Estudo de tecnologias para interfaces: Estudo de quais tecnologias são usadas como interface de usuário para controle da cadeira motorizada adaptada. Neste estudo são observados dispositivos de controle como Joystick e aplicativos que podem utilizar tecnologia Bluetooth ou VPN para comunicar com o Raspberry Pi.

      \item Escolha da linguagem de programação do aplicativo: Definição de qual linguagem será utilizada com base na problemática existente e nos recursos a serem utilizados. A linguagem escolhida para o desenvolvimento do aplicativo será definida conforme o resultado de estudo de utilziação de Bluetooth e VPN. Caso a escolha seja para dispositivos Android, então pode ser escolhida a linguagem nativa Java, caso a escolha seja para dispositivos iOS, então pode ser escolhida a linguagem nativa Objective-C ou Swift.

    \end{enumerate}

\section{Próximos passos}
  \subsection{Estrutura}
    \begin{enumerate}
      \item Fazer análise computacional das tensões equivalentes atuantes na estrutura, mediante critérios de falha de von Mises. Será utilizado o software ANSYS.
      \item Escolha de material mediante norma ABNT 6061 - T6.
    \end{enumerate}
  \subsection{Power Train}
    \begin{enumerate}
      \item NECESSARIO TERMINAR DE PEGAR AS INFORMAÇÕES DO GRUPO
      \item Escolha do motor quanto ao custo do mesmo no projeto. Motores escolhidos:
      \begin{enumerate}
        \item Zm 8070501 (12V 0,8 kW)
        \item Zm 8070502 (12V 0,8 kW)
        \item Zm 8010603 (12V 0,8 kW)
        \item Zm 8010604 (12V 0,8 kW)
        \item Zm 8010605 (12V 0,8 kW)
        \item Zm 8010606 (12V 0,8 kW)
        \item Zm 8010607 (12V 0,8 kW)
        \item Zm 8010608 (12V 0,8 kW)
      \end{enumerate}
    \end{enumerate}
  \subsection{Interface com Usuário}
    \begin{enumerate}
      \item Prova de conceito para comunicação VPN e Bluetooth para iOS;
      \item Prova de conceito para comunicação VPN e Bluetooth para Android;
      \item Prova de conceito para comunicação de Joystick ( cabo e Bluetooth);
    \end{enumerate}
  \subsection{Controle}
    \begin{enumerate}
     \item Definição de interface entre os componentes de controle;
     \item Versão inicial do controle com smartphone;
     \item Versão inicial do controle com joystick.
    \end{enumerate}
