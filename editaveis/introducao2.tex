\chapter[Introdução]{Introdução}

A Cadeira de rodas manual (CRM) é um importante instrumento para a funcionalidade diária na vida daqueles que tem os membros inferiores comprometidos. Segundo uma pesquisa realizada em 2010 pelo IBGE existe cerca de 4,4 milhões de indivíduos incapazes ou com grandes dificuldades de locomoção em todo o Brasil ~\cite{ibge:cartilha:2010}, cadeirantes, em sua maioria.

Segundo Sagawa et al  as CRM são consideradas meios de locomoção de baixa eficiência mecânica (2 a 10%), além disso os membros superiores não foram preparados para fazerem tantos esforços e movimentos repetidos,  para indivíduos que ainda estão em fase de adaptação esse esforço é ainda maior e para aqueles com sobrepeso os problemas de sobrecarga podem ser tão sérios quanto os riscos cardiovasculares.
Na busca de aumenta a funcionalidade de independência do individuo a cadeira de rodas elétrica surgiu oferecendo ao individuo maior facilidade e eficácia no deslocamento, no entanto, cadeiras de rodas elétricas têm um alto custo e muitas e não são portáteis como CRM.

Nesse âmbito, nos últimos anos projetos de automação de cadeiras de rodas manuais ~\cite{brunel:wheelchair:2004}, \cite{artigo_rudi}, \cite{patent_cadeira_rodas_eletrica},	 \cite{marcos:controle:2002}  vem sido testados, estes dão ao cadeirante a facilidade de mobilidade  e ao mesmo tempo utilizam-se do fato de que a cadeira de rodas manual do cadeirante já está com a ergometria  adaptada as necessidades do individuo.

Dessa forma, diante da importância entre a relação homem/cadeira de rodas o presente trabalho tem como objetivo prototipar um kit de automação de cadeira de rodas que seja acoplável a todas as cadeiras de rodas  Para garantir que esse produto seja compatível com o maior número de cadeiras de rodas do mercado possível, seguir-se-á o padrão especificado pela NBR 9050 (ABNT, 2004) e as dimensões do INMETRO~\cite{inmetro}

