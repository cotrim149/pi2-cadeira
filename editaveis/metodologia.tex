\chapter[Metodologia]{Metodologia}

\section{Estrutura}
Modelagem 3D do dispositivo: será construído um modelo 3D da estrutura, seguindo as dimensões especificadas pela NBR 9050 (ABNT, 2004) e as dimensões do IMETRO, utilizando para isto o software Catia V5 3D.

\section{Power Train}

\subsection{Motor}
O motor é uma máquina que tem a capacidade de transformar energia elétrica em energia mecânica (MEGGIOLARO 2011), existem dois tipos de motores, motor de corrente alternada (ca) e os de corrente contínua (cc). Será avaliado qual o tipo se encaixa melhor aos requisitos do projeto.

\subsection{Bateria}
Baterias são dispositivos que transformam energia química em elétrica e vice-versa. Por ser um processo reversível, as baterias podem ser carregadas e descarregadas várias vezes. Hoje no mercado existem vários tipos de baterias, com diferentes condições nominais, serão levantados os requisitos necessários e qual tipo de bateria se adéqua melhor ao projeto.

\subsection{Controle}
