\chapter[Planejamento Financeiro]{Planejamento Financeiro}
\subsection {Arrecadação}
Desde o início do projeto começamos a arrecadar um valor mensal, sem qualquer cálculo, usando apenas suposições. 
\itemInício da arrecadação = setembro
\itemQuantidade de meses de projeto = 4
\itemValor = 50,00
\itemQuantidade de integrantes = 13
\itemTotal a ser arrecadado = 2600,00
Com o amadurecimento e o conhecimento dos gastos reais do projeto esse valor se mostrou suficiente para custear o projeto.

\subsection {Planejamento de Gastos}
Um dos tópicos de grande importância para o projeto é o custo do produto que está sendo desenvolvido. Nessa sessão serão apresentados os valores levantados.\\
-Controle


\begin{center}
\begin{tabular}{ |c|c|c| } 
 \hline
Quantidade & Componente & Preço Unitário\\
 \hline
-Controle &  & \\
 1 & Raspberry Pi B+   &  R$ 250.00 \\ 
 4 & MSP430G2553  & R$ 20.00 \\ 
 1 & Joystick  & R$ 15.00 \\ 
 & Componentes Reserva & R$ 50.00\\
& SUBTOTAL & R$ 495.00\\
-Power train &  & \\
1& Baterias SLA  & R$ 300.00\\
2 & Moto-redutor&  R$ 560.00\\
2& RODA & R$ 75.00\\
& SUBTOTAL &  R$1570.00\\
-Estrutura &  & \\
1 &  Solda e material & R$600.00\\

 \hline
& TOTAL &  R$2663.00\\
 \hline
\end{tabular}
\end{center}

O custo total do projeto é de R$2663.00.  Vale ressaltar que apesar do custo do protótipo ser alto a nível industrial seria significantemente reduzido.
