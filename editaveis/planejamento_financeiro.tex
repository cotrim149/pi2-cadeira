\chapter[Planejamento Financeiro]{Planejamento Financeiro}

Um dos aspectos de grande relevância para o projeto é o custo. Nessa sessão faremos uma avaliação sobre o custo de fabricação de umm protótipo funcional de um kit para automação de cadeira de rodas eleétrica.

\section {Arrecadação}
Desde o início do projeto começamos a arrecadar um valor mensal, utilizando de uma estimativa inicial.

  \begin{itemize}
    \item Início da arrecadação = setembro
    \item Quantidade de meses de projeto = 4
    \item Valor = R\$50,00
    \item Quantidade de integrantes = 13
    \item Total a ser arrecadado = R\$2600,00
  \end{itemize}

Com o amadurecimento e o conhecimento dos gastos reais do projeto esse valor se mostrou suficiente para custear o projeto.

\section {Planejamento de Gastos}
Na tabela \ref{tab:custos_tabela}, serão apresentados os custos finais de fabricação do protótipo.

% Please add the following required packages to your document preamble:
% \usepackage{graphicx}
% \usepackage[normalem]{ulem}
% \useunder{\uline}{\ul}{}
\begin{longtable}{|c|c|c|}
\hline
Quantidade & Componente & Preço Unitário \\ \hline
Controle &  &  \\
1 & Raspberry Pi B+ & R\$ 250,00 \\
3 & MSP430G2553 & R\$ 20,00 \\
1 & Joystick & R\$ 15,00 \\
2 & Ponte H & R\$ 50,00 \\
  & Componentes Reserva & R\$ 50,00 \\
24 & Isolamento com mica e porca & R\$ 0,80 \\
5 & resistência 1k4 & R\$ 0,15 \\
3 & placa fenolíteo 25x25 & R\$18,00 \\
8 & Transistor IRF 3205 & R\$ 3,90 \\
4 & Capacitor eletrolítico 10 uF x 50 V & R\$ 0,30 \\
2 & Capacitor eletrolítico 1000uF x 25 V & R\$ 0,65 \\
2 & Capacitor eletrolítico 10uF x 100 V & R\$ 0,30 \\
4 & diodo in 4007 & R\$ 0,20 \\
4 & Transistor BC548 & R\$ 0,35 \\
4 & Transistor BC558 & R\$ 0,30 \\
2 & CI LM555 & R\$ 1,20 \\
6 & Resistência 100R & R\$ 0,20 \\
6 & Resistência 1k & R\$ 0,20 \\
2 & Resistência 2k2 & R\$ 0,20 \\
16 & Resistência 10k & R\$ 0,20 \\
2 & Resistência 15k & R\$ 0,20 \\
8 & Resistência 100k & R\$ 0,20 \\
16 & Parafuso & R\$ 0,60 \\
2 & Chicote de 2 vias & R\$ 6,50 \\
2 & 74LS08 & R\$ 4,90 \\
2 & 74LS04 & R\$ 4,00 \\
1 & Caixa para joystick & R\$ 7,50 \\
2 & RJ45 fêmea & R\$ 6,00 \\
24 & Borne com porca & R\$ 1,50 \\
24 & Plug Banana & R\$ 1,00 \\
4 & Fusível 15A & R\$ 0,50 \\
2 & Fusível 20A & R\$ 0,50 \\
2 & Fusível 25A & R\$ 0,50 \\
2 & Fusível 30A & R\$ 0,50 \\
2 & Espaguete termo retrátil 4,5 mm & R\$ 3,90 \\
1 & Fita isolante & R\$ 6,90 \\
1 & cabo flexível 2,5 mm, 4 metros Vermelho & R\$ 3,00 \\
1 & cabo flexível 2,5 mm, 2 metros Vermelho & R\$ 1,50 \\
1 & cabo flexível 2,5 mm, 2 metros azul & R\$ 3,00 \\
2 & cabinho flexível 2,5 mm, amarelo & R\$ 1,60 \\
2 & cabinho flexível 2,5 mm, preto & R\$ 1,60 \\
4 & Tubo de solda & R\$ 5,00 \\
 & Subtotal & R\$ 770,55 \\ \hline
\textit{Power Train} &  &  \\
1 & Baterias SLA & R\$ 300,00 \\
2 & Redutor MR com motor GPB 1:10 & R\$ 560,00 \\
2 & Rodas para anexo & R\$ 75,00 \\
 & Subtotal & R\$1.570,00 \\ \hline
Estrutura &  &  \\
1 & Solda e material & R\$300,00 \\
 & Subtotal & R\$300,00 \\ \hline
 & TOTAL & R\$2.640,55 \\ \hline
\caption{Tabela de gastos totais por subgrupo}
\label{tab:custos_tabela}
\end{longtable}

O custo total do projeto é de R\$ 2.640,55.  Vale ressaltar que apesar do custo do protótipo ser alto, a nível industrial seria significantemente reduzido.
