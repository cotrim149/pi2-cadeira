\chapter[Planejamento Financeiro]{Planejamento Financeiro}

Um dos aspectos de grande relevância para o projeto é o custo. Nessa sessão faremos uma avaliação sobre o custo para compararmos com o valor de mercado de uma cadeira de rodas elétrica

\section {Arrecadação}
Desde o início do projeto começamos a arrecadar um valor mensal, sem qualquer cálculo, usando apenas suposições.
  \begin{itemize}
    \item Início da arrecadação = setembro
    \item Quantidade de meses de projeto = 4
    \item Valor = R\$50,00
    \item Quantidade de integrantes = 13
    \item Total a ser arrecadado = R\$2600,00
  \end{itemize}
Com o amadurecimento e o conhecimento dos gastos reais do projeto esse valor se mostrou suficiente para custear o projeto.

\section {Planejamento de Gastos}

Um dos tópicos de grande importância para o projeto é o custo do produto que está sendo desenvolvido. Nessa sessão os valores são apresentados na tabela \ref{tab:custos_tabela}.

\begin{table}[!ht]
\centering
\begin{tabular}{ |c|c|c| }
 \hline
Quantidade & Componente & Preço Unitário\\
 \hline
Controle &  & \\
 1 & Raspberry Pi B+   &  R\$ 250.00 \\
 3 & MSP430G2553  & R\$ 20.00 \\
 1 & Joystick  & R\$ 15.00 \\
 2 & Ponte H &  R\$ 50.00 \\
  & Componentes Reserva & R\$ 50.00\\
& Subtotal & R\$ 475.00\\ \hline

Power train &  & \\
1 & Baterias SLA  & R\$ 300.00\\
2 & Redutor MR com motor GPB 1:10 &  R\$ 560.00\\
2 & Rodas para anexo & R\$ 75.00\\
& Subtotal &  R\$1570.00\\ \hline

Estrutura &  & \\
1 &  Solda e material & R\$600.00\\
 & Subtotal & R\$600.00 \\ \hline

& TOTAL &  R\$2645.00\\ \hline
\end{tabular}
\caption{Custos do projeto conforme áreas de trabalho}
\label{tab:custos_tabela}
\end{table}

O custo total do projeto é de R\$2663.00.  Vale ressaltar que apesar do custo do protótipo ser alto, a nível industrial seria significantemente reduzido.
