\chapter[Planejamento Financeiro]{Planejamento Financeiro}

Um dos aspectos de grande relevância para o projeto é o custo. Nessa sessão faremos uma avaliação sobre o custo para compararmos com o valor de mercado de uma cadeira de rodas elétrica

\section {Arrecadação}
Desde o início do projeto começamos a arrecadar um valor mensal, sem qualquer cálculo, usando apenas suposições.
  \begin{itemize}
    \item Início da arrecadação = setembro
    \item Quantidade de meses de projeto = 4
    \item Valor = R\$50,00
    \item Quantidade de integrantes = 13
    \item Total a ser arrecadado = R\$2600,00
  \end{itemize}
Com o amadurecimento e o conhecimento dos gastos reais do projeto esse valor se mostrou suficiente para custear o projeto.

\section {Planejamento de Gastos}

Um dos tópicos de grande importância para o projeto é o custo do produto que está sendo desenvolvido. Nessa sessão os valores são apresentados na tabela \ref{tab:custos_tabela}.

% Please add the following required packages to your document preamble:
% \usepackage{graphicx}
% \usepackage[normalem]{ulem}
% \useunder{\uline}{\ul}{}
\begin{table}[!ht]
\centering
\resizebox{\textwidth}{!}{%
\begin{tabular}{|c|c|c|}
\hline
Quantidade & Componente & Preço Unitário \\ \hline
Controle &  &  \\
1 & Raspberry Pi B+ & R\$ 250.00 \\
3 & MSP430G2553 & R\$ 20.00 \\
1 & Joystick & R\$ 15.00 \\
2 & Ponte H & R\$ 50.00 \\
 & Componentes Reserva & R\$ 50.00 \\
24 & Isolamento c/ mica e porca & 0,8 \\
5 & resistência 1k4 & 0,15 \\
3 & placa fenolíteo 25x25 & 18 \\
14 & Transistores IRF3205 & 3,9 \\
8 & Capacitor eletrolítico 10 uF x 50 V & 0,3 \\
4 & Capacitor eletrolítico 1000uF x 25 V & 0,65 \\
4 & Capacitor eletrolítico 10uF x 100 V & 0,3 \\
8 & diodo in 4007 & 0,2 \\
8 & Transistor bc 548 & 0,35 \\
8 & Transistor bc 558 & 0,3 \\
5 & lm 555 & 1,2 \\
8 & Resistência 100R & 0,2 \\
12 & resistência 1k & 0,2 \\
4 & Resistência 2k2 & 0,2 \\
32 & Resistência 10k & 0,2 \\
4 & Resistência 15k & 0,2 \\
4 & Bornes & 2,5 \\
8 & Resistência 100k & 0,2 \\
16 & Parafuso & 0,6 \\
2 & Chicote de 2 vias & 6,5 \\
2 & 74LS08 & 4,90 \\
2 & 74LS04 & 4 \\
1 & cx & 7,5 \\
2 & RJ45 fêmea & 6 \\
10 & Borne c/ porca & 1,5 \\
10 & Plug Banana & 1 \\
10 & Fusível 5A & 0,5 \\
4 & Fusível 10A & 0,5 \\
2 & Fusível 20A & 0,5 \\
2 & Fusível 25A & 0,5 \\
2 & Espaguete termo retrátil 4,5 mm & 3,9 \\
1 & Fita isolante & 6,9 \\
1 & cabo flexível 2,5 mm, 4 metros Vermelho & 3 \\
1 & cabo flexível 2,5 mm, 2 metros Vermelho & 1,5 \\
1 & cabo flexível 2,5 mm, 2 metros azul & 3 \\
2 & cabinho flexível 2,5 mm, amarelo & 1,6 \\
2 & cabinho flexível 2,5 mm, preto & 1,60 \\
1 & Tubo de solda & 5 \\
 & Subtotal & R\$ 475.00 \\ \hline
Power train &  &  \\
1 & Baterias SLA & R\$ 300.00 \\
2 & Redutor MR com motor GPB 1:10 & R\$ 560.00 \\
2 & Rodas para anexo & R\$ 75.00 \\
 & Subtotal & R\$1570.00 \\ \hline
Estrutura &  &  \\
1 & Solda e material & R\$300.00 \\
 & Subtotal & R\$300.00 \\ \hline
 & TOTAL & R\$2345.00 \\ \hline
\end{tabular}
}
\caption{My caption}
\label{tab:custos_tabela}
\end{table}

O custo total do projeto é de R\$2663.00.  Vale ressaltar que apesar do custo do protótipo ser alto, a nível industrial seria significantemente reduzido.
